\documentclass[a4paper]{article}

\usepackage[utf8]{inputenc}
\usepackage[T1]{fontenc}
\usepackage{textcomp}
\usepackage[dutch]{babel}
\usepackage{amsmath, amssymb}
\usepackage{ntheorem}
\usepackage{titlesec}

% figure support
\usepackage{import}
\usepackage{xifthen}
\pdfminorversion=7
\usepackage{pdfpages}
\usepackage{transparent}
\newcommand{\incfig}[1]{%
	\def\svgwidth{\columnwidth}
	\import{./figures/}{#1.pdf_tex}
}

%----------Theorems----------%

\newcounter{counter}
\numberwithin{counter}{subsection}

\theoremstyle{break}
\theoremindent20pt 
\theorembodyfont{\normalfont}
\theoremheaderfont{\normalfont\bfseries\hspace{-\theoremindent}}
\newtheorem{definition}{Definition}
\newtheorem{exmp}[counter]{Example}
\newtheorem{lemma}[counter]{Lemma}
\newtheorem*{solution}{Solution}
\newtheorem{exe}[counter]{Exercise}
\newtheorem{theorem}[counter]{Theorem}
%------Miscellaneous---------%

\titleformat{\section}[block]{\Large\bfseries\filcenter\underline}{}{1em}{}
\newcommand{\sectionbreak}{\clearpage}

\pdfsuppresswarningpagegroup=1

\begin{document}

\section{Review}%
\label{sec:review}

\begin{exmp}
	Suppose we want to make a 3-digit number using the digits in $\Omega = \{1,2,3,4\}$. \\
	To form this number we will make an arrangement using elements from $\Omega$. \\
	We might make the number 231, a particular arrangement of the subset $A = \{1,2,3\} \subset \Omega$. A different permeation of A generates a new number, and a different arrangement. 
\end{exmp}

\newtheorem{sol}{Solution}
\begin{exmp}
	Given a classical deck of 52 cards, how many poker hands (set of 5 cards) are in the category three of a kind? (that is, no better than three of a kind)

	\begin{sol}
		Two possible ways:
		\begin{enumerate}
			\item build an unordered hand.
			\item build an ordered hand and divide by 5! to remove the over counting
		\end{enumerate}

		\textbf{First Way:} 
		\begin{enumerate}
			\item Consider $\Omega = \{x_1, x_2, x_3, x_4, x_5\}$ with $x_i$ a card from the deck
			\item Consider building all possible hands with 3 of a kind, denote this subset A
			\item to build A
				\begin{enumerate}
					\item Choose the rank of three of a kind (13 choices).
					\item Pick 3 cards out of the possible 4 suits of this rank, $\begin{pmatrix} 4 \\ 3 \end{pmatrix}$ choices.
					\item Choose 2 other ranks for the two reaming cards $\begin{pmatrix} 12 \\ 2 \end{pmatrix}$.
					\item Choose one of the suit for both of these cards (4 choices for each so $4^2$ for both).
				\end{enumerate}
			\item in total $\#A = 13 * \begin{pmatrix} 4 \\ 3 \end{pmatrix}* \begin{pmatrix} 12 \\ 2 \end{pmatrix} * 4^2 = 54912$
		\end{enumerate}

		\textbf{Second way:} 
		\begin{enumerate}
			\item Consider $\Omega = \{x_1, x_2, x_3, x_4, x_5\}$ with $x_i$ a card from the deck, all different.
			\item Consider building all possible ordered hands with 3 of a kind denote this subset $A$ 
			\item to build A:
			\begin{enumerate}
				\item First choose three slots for the three of a kind $\begin{pmatrix} 5 \\ 3 \end{pmatrix}$ for this choice
				\item Assign 1 card to one of these slots (52 choices)		
				\item Assign a second card to one of these slots with the same rank (3 choices)
				\item Assign a third card to the last slot with the same rank (2 choices)
				\item Assign one card different than the other three (48 choices)  
				\item Assign one card different to all other choices (44 choices)
			\end{enumerate}

			\item In total $\#A = \begin{pmatrix} 5 \\ 3 \end{pmatrix} * 52 * 3 * 2 * 48 * 44$
			\item divide by 5! to only consider the possible sets, i.e the solution is 54912.
		\end{enumerate}
	\end{sol}
\end{exmp}

\section{Sample space and events}%
\label{sec:sample_space_and_events}

\begin{definition}
	A sample space $\Omega$	is the set of all possible outcomes of an experiment.
\end{definition}

\begin{exmp}
	(i) 3 coin tosses, One die roll, (iii) Sum of two	rolls, seconds waiting for bus, number of attempts to win a lottery.  
	\begin{enumerate}
		\item $S = \{HHH, HHT, HTH, HTT, T H H, THT, TTH, TTT\} $
			\item $S = \{1,2,3,4,5,6\} $
	\end{enumerate}
\end{exmp}

\begin{exe}
    \textbf{Q:} Is playing five rounds of Russian roulette is this the same as coin flipping.\\
    \textbf{A:} No it is not the same because $\Omega = \{D, LD, LLD, LLLD, LLLLD, LLLLLL\}$\\
    \textbf{Q:} Sequencing three nucleotides (each yuletide is A, C, g or T).\\
	$\Omega = \{AAA, CCC, GGG, TTT, AAC AAT, AAG, \ldots \}$. What is the cardinally of the space?\\
	\textbf{A:} $\#\Omega = |\Omega|= 4^3 = 64$
\end{exe}

\begin{definition}[Event]
	An event E is any subset of the sample space $\Omega$.
\end{definition}

\begin{exmp}
    2 heads out of three flips ($E = \{HHT, HTH, THH\}$), time is < 2 minutes when waiting for the bus, $E = [0,12)$.
\end{exmp}

\subsection{Collection of events}%
\label{sub:collection_of_events}

In the following, we consider possibilities associated ta a collection of events denoted $\mathcal{F}$.
\begin{itemize}
	\item If $\Omega$ is discrete (i.e countable) we will simply consider. \[
	\mathcal{F} = 2^{\Omega}
	.\] 
	the power set (the set of all subsets of $\Omega$) 
\item If $\Omega$ is continuous (i.e. Uncountable) like $\mathbb{R}$, the power set of $\Omega$ is too complex.
	We restrict ourselves to events that are intersections or unions of intervals $[a, +\infty)$
	\item Formally the collection of events of interest must be $\sigma$-algebra
\end{itemize}

\newtheorem{note}{Note}
\begin{note}
	In all cases: \O $\in \mathcal{F}, \: \Omega \in \mathcal{F}$
\end{note}

\newpage
\subsection{Set operations}%
\label{sub:set_operations}

Let $A, B$ be two events.
\begin{definition}
	\begin{enumerate}
		\item Intersection: A $\cap$ B: both A and B occur
		\item Union: A $\cup $ B: at least one of A or B occur
		\item The complement of A, $A^{C}$ : A does not occur
		\item A $\subset $ B: occurrence of A impels occurrence of B.
		\item Set difference A \ B := $A \cap (B^{c)}:$ A occurs, B does not.
	\end{enumerate}
\end{definition}

\begin{definition}[Disjoint Sets]
	Two events A and B are mutually exclusive or disjoint if they have no outcomes in common, i.e. $A\cap B = $ \O.
\end{definition}

\begin{exmp}
	Singletons( one outcome )  of distinct elements are always disjoint. So for a die, rolling a 6 (A = {6}) and rolling a 4 (B = {4}) are mutually disjoint set events. 
\end{exmp}

\begin{exe}
	Let A be the event that a person is male, B that person is under 30, and C that the person speaks French. Describe in symbols (feel free to use a Venn diagram to help):
	\begin{itemize}
		\item A male at least 30 year old
			\item A female under 30 who speaks French
				\item A male who is either under 30 or who speaks french 
	\end{itemize}

	\begin{sol}
		\begin{itemize}
			\item $A \cap B^{c}$
			\item $A^{c} \cap B \cap C$
			\item $(A \cap B) \cup (A \cap C)$
		\end{itemize}	
	\end{sol}
\end{exe}

\begin{exe}
	You roll 2 dice, what is the cardinality of the event:\\
	E = "the sum larger than or equal to 10"? By cardinality it means the number of possible outcomes that result in this event.
	\begin{sol}
		The event is defined as: \[
			A = \{(4,6), (5,5)(5,6), (6,6),(6,5),(6,4)\} 
		.\] 
		So the cardinality is equal to 6
	\end{sol}
\end{exe}

\section{Axioms of Probability}% 
\label{sec:axioms_of_probability}

\begin{definition}[informal]
	Given an experiment, a sample space $\Omega$, and all possible events $\mathcal{F}$\footnote{If $\Omega$ is discrete, $\mathcal{F}= 2^{\Omega}$, otherwise see next lecture for rigors definition} to define a probability measure is to associate each event $A \in \mathcal{F}$ with a number $\mathcal{P(A)}$ between 0 and 1, called the probability of the event.\\ 
\end{definition}

There is no agreement in how probabilities should be interpreted.
\begin{itemize}
	\item \textbf{Frequentest} interpretation: The probability of an event A is the proportion of times (frequency) that A occurs in an infinite sequence  (or very long run) of separate tries of the experiment. \[
		\mathcal{P}(A) = \lim_{n \to \infty} \frac{\text{\# times it happens}}{n}
	.\]  

\item \textbf{Bayesian} interpretation: The probability of event A reflects a (usually subjective) belief on the likelihood of A.
\end{itemize}

\begin{definition}[Probablity Measure]
	Let $\Omega$ be the sample space, and $\mathcal{F}$ be the set of all possible events. The probability measure (also called probability distribution or simply probability ) is a function from $\mathcal{F}$ into the real numbers such that:

	\begin{enumerate}
		\item $0 \le \mathcal{P}(A) \le 1$ for any event A
		\item $\mathcal{P}(\Omega) = 1$ 
		\item For $A_1, A_2, \ldots$ any sequence of pairwise disjoint events\\ $(A_i \cap A_j = $ \O for $i \neq j)$, \[
				\mathcal{P}(\bigcup_{i = 1}^{\infty} A_i) = \sum_{i=1}^{\infty} \mathcal{P}(A_i)
		.\] 
	\end{enumerate}

The triplet ($\Omega$, $\mathcal{F}$, $\mathcal{P}$) is called \textbf{probability space}.
\end{definition}

\subsection{Immediate consequences}%
\label{sub:immediate_consequences}

\begin{itemize}
	\item Finite additivity: For disjoint $A_1,A_2,\ldots,A_m \in \mathcal{F}$, \[
			\mathcal{P}(\bigcap_{i=1}^{m} A_i) = \mathcal{P}(A_1) + \ldots + \mathcal{P}(A_m)
	.\] 

\item For a finite set $\Omega = \{\omega_1,\ldots, \omega_N\}$, \[
		1 = \mathcal{P}(\Omega) = \mathcal{P}({\omega_1}) + \ldots + \mathcal{P}({\omega_N})
.\] 

i.e the probabilities of the singletons must sum up to one
\end{itemize}

\begin{definition}[Equally Likely Probablites]
	Suppose $\Omega$ is finite and consist of N elements: \[
	\Omega = \{\omega_1, \omega_2, \ldots, \omega_N\} 
	.\] 
	A probability $\mathcal{P}$ is called equally likely if $\mathcal{P}(\omega_1) = \ldots = \mathcal{P}(\omega_N) = \frac{1}{N}$.\\
	Clearly, $\mathcal{P}$(\O) = 0 and $\mathcal{P}(\Omega) = 1 $ and for any event $E \in \mathcal{F}$, \[
		\mathcal{P}(E) = \frac{\#E}{N}
	.\] 
\end{definition}

\begin{exmp}
	Toss a fair coin twice. $\Omega = \{H H, T T, H T, T H\} $.\\
	For $A = \{H T, T T\}, \: \mathcal{P}(A) = \frac{2}{4} = \frac{1}{2}$
\end{exmp}

\begin{exmp}
	An unfair die would be a die such that probabilities of each outcome is modified.  
\end{exmp}

\begin{exe}
	\begin{enumerate}
		\item Consider a fair die, such that the probability of rolling each face is equally probable. What is the probability of having an even number?
		\item Consider two fair dice. What is the probability that the sum of the rolled faces is 8?
	\end{enumerate}
\end{exe}

\begin{sol}
    \begin{enumerate}
        \item \begin{itemize}
            \item The probability of having an even number out of the die would be the set $A = \{2,4,5\} $ and the prob. is $\mathcal{P}(A) = \frac{1}{2}$  
        \end{itemize}
    \item \begin{itemize}
        \item The sample space is $\Omega = \{(i,j) : i, j \in {1, \ldots, 6}\} $
        \item Here we have tuples (ordered pairs). The dice are fair so outcomes in $\Omega$ are equally probable, i.e., $\mathcal{P}({(i,j)}) = \frac{1}{36}$.
        \item The even of interest is: \[
                D = {\text{the sum of two dice is 9}} = \{((2,6), (3,5), (4,4), (5,3), (6,2)\} 
            .\] 
        \item There for $\mathcal{P}(D) = \frac{5}{36}.$ We have order pairs here we could roll a 2 and then 6. Or a 6 or a 2
    \end{itemize}
    \end{enumerate}
\end{sol}

\begin{exe}
    \begin{enumerate}
		\item Consider flipping a fair coin three times. What is the prob. to have at least two heads in a row?
		\item You roll 2 fair dice, what is the probability to have a sum larger or equal than 10
    \end{enumerate}
\end{exe}
\begin{sol}

    \begin{enumerate}
        \item \begin{itemize}
            \item Rather than using {H,T} we will use (0,1). The sample space here is: \[
                    \Omega = \{(i,j, k: i,j,k \in \{0,1\} \} 
            .\] 
        \item The coin is fair so all outcomes $\Omega$	are equal i.e., $\mathcal{P}({i,j,k}) = \frac{1}{2^3}$. The even of interest is:
        \item D = {at least two heads in a row} = $\{\{0,0,1\},\{1,0,0\}, \{0,0,0\} \}$
        \item Therefore $\mathcal{P}(D) = \frac{3}{8}$
        \end{itemize}
            
    \item \begin{itemize}
        \item The event is: \[
        A = \{\{4,6\}, \{5,5\}, \{5,6\}, \{6,6\}, \{6,5\}, \{6,4\} \} 
        .\] 

        All events are equal, which means there are $6^2$ possible events. \\
        So the Prob(A) is $\frac{6}{36} = \frac{1}{6}$
        \end{itemize}
    \end{enumerate}
\end{sol}

	\section{Sampling from an urn}%
	
	\subsection{Ordered sampling with replacement}%
	
	\begin{definition}
		Consider an urn with $N \ge 2$ balls lobed as $\{1,2,\ldots,n\} $. Pick one ball without looking, note its label, and put it back. Repeat this k times. The outcome is a k-tuple $\{a_1,a_2,..,a_k\} $ \[
            \Omega = {\text{all k-tuples of } \{1,\ldots,N\}}, \#\Omega = N^{k}
		.\] 
	\end{definition}
	
	\begin{exmp}
	Suppose our urn contains 5 balls labeled 1,2,3,4,5. Sample 3 balls with replacement and produce an ordered list of the numbers drawn. 
	The sample space is: \[
        \Omega = \{1,2,3,4,5\}^{3}= \{\{s_1,s_2,s_3\}: \text{each } s_i \in \{1,2,3,4,5\} \}, \: \Omega = 5^3=125
	.\]  
	We have for example \[
        \mathcal{P}(\text{the sample is } (2,1,5)) = \mathcal{P}(\text{the sample is } (2,2,3)) = \frac{1}{125}
	.\] 
	\end{exmp}

	\begin{exmp}
		Repeated flips of a coin or roll of a die are also sampling with replacements from the set {H,T} or {1,2,3,4,5,6}
	\end{exmp}

\subsection{Ordered sampling without replacement}%

Pick one ball without looking, note its label, but dont put it back. Repeat this k times. The outcome is a k-tuple without repeats, i.e., \[
    \Omega = {\text{all k-arrangments of 1},\ldots,N}
.\]   

\begin{exmp}
	Consider again the urn with 5 balls labled 1,2,3,4,5. Sample 3 balls without replacement. \[
        \Omega = \{\{s_1,s_2,s_3\} : \text{each } s_i \in \{1,2,3,4,5\} \text{ and } s_1,s_2,s_3, \text{are all distinct}\} 
	.\]  

	We have 5 choices for the first, 4 for the second, 3 choices for the third so $\#\Omega = 5 * 4 * 3 = (5)_3 $ and \[
        \mathcal{P}(\text{the sample is } (2,1,5)) = \frac{1}{5 * 4 * 3} = \frac{1}{60}	
	.\] 
\end{exmp}

\subsection{Unordered sampling without replacement}%

\begin{definition}
	Unordered sampling without replacement Pick one ball without looking, note its label but don't put it back. Repeat this k times. This time the order of the balls does not matter.\\
	Namely, consider the outcome to be a k-subset \[
        \Omega = \{\text{all k-subset of 1}, \ldots, N\}, \: \#\Omega = \begin{pmatrix} N \\ k \end{pmatrix}
	.\] 

\end{definition}

\begin{exe}
	Class of 24 Children. All picking is done uniformly at random.
	\begin{enumerate}
		\item Each day one child leads the class to lunch. What is the prob. That Alex is chosen on Monday and Wed. And Julie is picked on Tuesday.
		\item One student chosen to be president, one vice, one treasurer. They cannot hold more than one postion. Prob that Mary is pres, Cory is vice, and Matt treasurer.
		\item Same but now what is the prob that Ben is either president or vice.
		\item A team of 3 students is chosen at random. What is the prob that the team consist in Shane, heather, and Laura?
		\item A team of 3 is chosen. What is the prob that Mary is on the team?
	\end{enumerate}

	\begin{sol}
		\begin{enumerate}
			\item \begin{itemize}
                \item $\mathcal{P}(\text{Alex, Julie, Alex}) = 24^{-3} = \frac{1}{24^3}$
			\end{itemize}
		\item \begin{itemize}
            \item $\mathcal{P}(\text{Marry pres, Cory vice, Mat treas}) = \frac{1}{24 * 23 * 22}$
		\end{itemize}
	\item \begin{itemize}
        \item $\mathcal{P}(\text{Ben pres}) = \frac{1*23*22}{24*23*22} = \mathcal{P}(\text{Ben vice pres})$
        \item $\mathcal{P}(\text{Ben Pres}) \cup \mathcal{P}(\text{Ben vice}) = \mathcal{P}(pres) + \mathcal{P}(vp)$ 
	\end{itemize}
		\item \begin{itemize}
			\item $\mathcal{P}(\text{Shane, Heather, Laura}) = \frac{1}{\begin{pmatrix} 24 \\ 3 \end{pmatrix}} = \frac{1}{\frac{24!}{3!(21)!}}= \frac{1}{2024}$
		\end{itemize}
		\item \begin{itemize}
            \item $\begin{pmatrix} 23 \\ 2 \end{pmatrix}$ teams include Mary so $\mathcal{P}(\text{Mary is in the team}) = \frac{\begin{pmatrix} 23 \\ 2 \end{pmatrix}}{\begin{pmatrix} 24 \\ 3 \end{pmatrix}} $
		\end{itemize}
	\end{enumerate}
	\end{sol}
\end{exe}
\end{document}
